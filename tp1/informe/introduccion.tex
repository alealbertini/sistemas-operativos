\section{Introducción}

En el presente trabajo nos proponemos a implementar la lógica de varios planificadores de tareas.

Nuestro trabajo consistir\'a en implementar, utilizando las estructuras provistas por la c\'atedra, ciertos algoritmos de \textit{scheduling} en lenguaje c++. Realizaremos experimentos empíricos e intentaremos hallar el \textit{quantum} \'optimo para el algoritmo de \textit{Round Robin}, con respecto a varias m\'etricas. Estas m\'etricas podr\'ian implicar criterios contradictorios, por lo que el objetivo ser\'a encontrar un balance adecuado entre ellas.

Además analizaremos un paper sobre scheduling basados en prioridades para procesos en tiempo real con ciertos requerimientos de deadlines obligatorios. Implementaremos estos últimos y buscaremos ejemplos para ilustrar el comportamiento de estos.

El trabajo se encuentra dividido en 10 ejercicios cuya meta de estos es abarcar lo anteriormente comentado. Se decidio omitir una secci\'on resultados agregando los gráficos directamente en el desarrollo. Lo que se busca con esta estructuración del informe es mantener cada ejercicio con explicaciones y gr\'aficos juntos para comodidad del lector.

