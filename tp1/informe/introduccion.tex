\section{Introducción}

En el presente trabajo nos proponemos a implementar la lógica de de varios planificadores de tareas.

Para implementar los algoritmos utilizaremos el simulador provisto por la catedra, nuestro trabajo consistira en implementar ciertos algoritmos de scheduler en lenguaje c++ y realizar experimentos empíricos tratando de encontrar en particular para Round Robin el Quantum que nos de mejores resultados en las pruebas tratando de satisfacer métricas que nos son compatibles entre sí (Es decir encontrar un punto de equilibrio entre ellas).

Además analizaremos un paper sobre scheduling basados en prioridades para procesos en tiempo real con ciertos requerimientos de deadlines obligatorios. Implementaremos estos últimos y buscaremos ejemplos para ilustrar el comportamiento de estos.

La trabajo se encuentra dividido en 10 ejercicios cuya meta de estos es abarcar lo anteriormente comentado. Se decidio omitir una sesión resultados agregando los gráficos directamente en la sesión desarrollo, lo que se busca con esta estructuración del informe es mantener cada ejercicio con explicaciones y graficos juntos para comodidad del lector.

