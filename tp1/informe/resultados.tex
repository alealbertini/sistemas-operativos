\section{Resultados}

\subsubsection*{Experimentos sobre el grado del polinomio interpolador}

\includegraphics[scale=0.5]{graficos/gradoPol.png}

Para el experimento sobre el grado del polinomio se corrieron todas las
instancias de las carpetas rectas, cuadráticas y cubicas provistas por la
cátedra.  Cada una de estas instancias se corrió para cada grado del polinomio,
por lo tanto se probaron 15 instancias por grado variando el grado de 1 a 8.

\begin{verbatim}
Cantidad de aciertos sobre 15 instancias
grado 1: 8 
grado 2: 10
grado 3: 13
grado 4: 13
grado 5: 13
grado 6: 10
grado 7: 9
grado 8: 8
\end{verbatim}


Los resultados se encuentran en el archivo visualizador/resultados/testGradoPol.out

\subsubsection*{Experimentos sobre el tamaño del historial}

\includegraphics[scale=0.5]{graficos/kPuntos.png}

Para el experimento del historial se utilizaron las siguientes instancias:

\begin{itemize}
  \item Curvas no atajas por el arquero anterior:
  \begin{itemize}
    \item cl.tiro
    \item c3.tiro
  \end{itemize}
 \item Curvas atajadas por el arquero anterior:
  \begin{itemize}
    \item c2.tiro
    \item r1.tiro
    \item cb1.tiro
  \end{itemize}
  \item Trayectos que no describen curvas exactas no probadas en el arquero anterior: 
  \begin{itemize}
   \item 1.tiro
   \item 2.tiro
   \item 3.tiro
   \item 4.tiro
   \end{itemize}
\end{itemize}

Se considera k=25 como el polinomio armado con todos los puntos de la
instancia, ya que ninguna instancia supera este numero de puntos.  Los
resultados son:

\begin{itemize}
\item Para K=25:
  \begin{itemize}
    \item Atajo: c2.tiro, r1.tiro, cb1.tiro,3.tiro,4.tiro
    \item No atajo: c1.tiro, c3.tiro, 1.tiro, 2.tiro
  \end{itemize}
\item Para K$\neq$25:
  \begin{itemize}
    \item Atajo: c2.tiro(exceptuando k=5), r1.tiro, cb1.tiro,3.tiro,4.tiro
    \item No atajo: c1.tiro, c3.tiro, 1.tiro, 2.tiro
  \end{itemize}
\end{itemize}

Los resultados se encuentran en el archivo visualizador/resultados/testK.out

Dado que las pruebas no mejoran ni empeoran el rendimiento de arquero al variar
este parámetro decidimos no limitar la cantidad de puntos tomados para nuestra
implementación final del robot.

\subsubsection*{Experimentos sobre las tácticas del arquero}

\includegraphics[scale=0.5]{graficos/tacticas.png}

En este grafico 0 significa sin tácticas, 1 significa solo táctica 1 y
2 solo táctica 2.

Para el experimento sobre las tácticas se utilizaron las instancias contenidas
en las carpetas fancy, instancia2 e instancias3.  Las parámetros se realizaron
con los parámetros $g=3$ y $k$ sin acotar.  Los resultados se encuentran en el
archivo visualizador/resultados/tac1ytac2.out


\includegraphics[scale=0.5]{graficos/grado_pol.png}

\includegraphics[scale=0.5]{graficos/grado_con_ruido30.png}
\includegraphics[scale=0.5]{graficos/grado_con_ruido70.png}

\includegraphics[scale=0.5]{graficos/cbp30_noRuido.png}
\includegraphics[scale=0.5]{graficos/cbp30_Ruido.png}
\includegraphics[scale=0.5]{graficos/cbp70_Ruido.png}
\includegraphics[scale=0.5]{graficos/cbp70_noRuido.png}
\includegraphics[scale=0.5]{graficos/rp70_noRuido.png}
\includegraphics[scale=0.5]{graficos/rp70_Ruido.png}

